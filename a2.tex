\documentclass{article}
\usepackage{fullpage,amsmath,amssymb}
\usepackage{hyperref}
\usepackage{calc}  % arithmetic in length parameters
\usepackage{enumitem}  % more control over list formatting
\usepackage{fancyhdr}  % simpler headers and footers
\usepackage[margin=1in]{geometry}  % page layout
\usepackage{lastpage}  % for last page number
\usepackage{relsize}  % easier font size changes
\usepackage[normalem]{ulem}  % smarter underlining
\usepackage{url}  % verb-like typesetting of URLs
\usepackage{xfrac}  % nicer looking simple fractions for text and math

%\usepackage[T1]{fontenc}  % use true 8-bit fonts
%\usepackage{slantsc}  % allow slanted small-caps
%\usepackage{microtype}  % perform various font optimizations
%% Use Palatino-based monospace instead of kpfonts' default.
%\usepackage{newpxtext}

% Common macros.
\input{./macros}
\newcommand*\st{\mathrel{|}}  % "such that" for set extension


\title{CSC236H, Fall 2018\\
Assignment 2\\
}
\author{Ruizhe Lu}
\renewcommand{\today}{~}
\hypersetup{pdfpagemode=Fullscreen,
  colorlinks=true,
  linkfileprefix={}}

\begin{document}
\maketitle

%begin questions
\begin{enumerate}

  %(1)
  \item 
    \begin{enumerate}
  
    % 0: *     
    % 1: (**)    
    % 2: ((**)*)   (*(**))     
    % 3: ((**)(**))    (*((**)*))  (((**)*)*)   (*(*(**)))  ((*(**))*)
    % 4: 
    \item   \strong{Solution:} 
            \begin{enumerate} [label=(\alph*)]
            % base
            \item 1

            % 0 + 0
            % 2c(0)
            \item 1

            % +4

            % 1+1 || 0+2
            \item 5

            % +9

            % 1+2 || 0+3
            % 4      10
            \item 14
            \end{enumerate}

    \item   \strong{Solution:}
    % guess
            \begin{equation}
                c(n) = \begin{cases}
                    1,  & n = 0 \\
                    1,  & n = 1 \\
                    c(n-1) + (n-1)^2,   & n > 1
                \end{cases}
            % why?
    \end{equation}
  \end{enumerate}



%(2)
\item   
  \begin{enumerate}

    %(a)
    \item \strong{Solution:}
    % 3 4 5

    %base
    % n=3: 1
    % n=4: 1
    % n=5: 1

    % n=6: 3+3 1

    %(b)
    \item \strong{Solution:}



\end{enumerate} 

%(3)
\item
\begin{enumerate}
  %(a)
  \item \strong{Proof:}
  %(b)
  \item \strong{Proof:}
  %(c)
  \item \strong{Proof:}
\end{enumerate} 

%(4)
\item
\begin{enumerate}
  %(a)
  \item \strong{Proof:}
  %(b)
  \item \strong{Proof:}
\end{enumerate}




\end{enumerate}
\end{document}
